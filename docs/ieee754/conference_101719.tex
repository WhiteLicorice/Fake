\documentclass[conference]{IEEEtran}

\makeatletter

\def\ps@IEEEtitlepagestyle{%
  \def\@oddfoot{\mycopyrightnotice}%
  \def\@evenfoot{}%
}
\def\mycopyrightnotice{%
  {\footnotesize XXX-X-XXXX-XXXX-X/XX/\$XX.00~\copyright~20XX IEEE\hfill}% <--- Change here
  \gdef\mycopyrightnotice{}
}


\usepackage{blindtext}
\usepackage{eso-pic}
\IEEEoverridecommandlockouts
% The preceding line is only needed to identify funding in the first footnote. If that is unneeded, please comment it out.
\usepackage{cite}
\usepackage{amsmath,amssymb,amsfonts}
\usepackage{algorithmic}
\usepackage{graphicx}
\usepackage{textcomp}
\usepackage{xcolor}
\def\BibTeX{{\rm B\kern-.05em{\sc i\kern-.025em b}\kern-.08em
    T\kern-.1667em\lower.7ex\hbox{E}\kern-.125emX}}
    
\usepackage{eso-pic}
\newcommand\AtPageUpperMyright[1]{\AtPageUpperLeft{%
 \put(\LenToUnit{0.17\paperwidth},\LenToUnit{-2cm}){%
     \parbox{0.9\textwidth}{\raggedleft\fontsize{8}{11}\selectfont #1}}%
 }}%
\newcommand{\conf}[1]{%
\AddToShipoutPictureBG*{%
\AtPageUpperMyright{#1}
}
}    
    
    
\begin{document}
\title{\vspace*{1cm} FaKe: Filipino Language Fake News Detection
System Using Machine Learning\\
%{\footnotesize \textsuperscript{*}Note: Sub-titles are not captured in Xplore and should not be used}
%\thanks{Identify applicable funding agency here. If none, delete this.}
}

\author{\IEEEauthorblockN{1\textsuperscript{st} Rene Andre Jocsing}
\IEEEauthorblockA{\textit{Division of Physical} \\
\textit{Sciences and Mathematics,} \\
\textit{College of Arts and Sciences} \\
\textit{University of the Philippines Visayas}\\
Iloilo, Philippines \\
rbjocsing@up.edu.ph}
\and
\IEEEauthorblockN{2\textsuperscript{nd} Chancy Ponce de Leon}
\IEEEauthorblockA{\textit{Division of Physical} \\
\textit{Sciences and Mathematics,} \\
\textit{College of Arts and Sciences}\\
\textit{University of the Philippines Visayas}\\
Iloilo, Philippines \\
cmponcedeleon@up.edu.ph}
\and
\IEEEauthorblockN{3\textsuperscript{rd} Coebe Austin Lupac}
\IEEEauthorblockA{\textit{Division of Physical} \\
\textit{Sciences and Mathematics,} \\
\textit{College of Arts and Sciences} \\
\textit{University of the Philippines Visayas}\\
Iloilo, Philippines \\
cvlupac1@up.edu.ph}
\and
\IEEEauthorblockN{4\textsuperscript{th} Francis Dimzon}
\IEEEauthorblockA{\textit{Division of Physical} \\
\textit{Sciences and Mathematics,} \\
\textit{College of Arts and Sciences}\\
\textit{University of the Philippines Visayas}\\
Iloilo, Philippines \\
fddimzon1@up.edu.ph}
% \and
% \IEEEauthorblockN{5\textsuperscript{th} Given Name Surname}
% \IEEEauthorblockA{\textit{dept. name of organization (of Aff.)} \\
% \textit{name of organization (of Aff.)}\\
% City, Country \\
% email address or ORCID}
% \and
% \IEEEauthorblockN{6\textsuperscript{th} Given Name Surname}
% \IEEEauthorblockA{\textit{dept. name of organization (of Aff.)} \\
% \textit{name of organization (of Aff.)}\\
% City, Country \\
% email address or ORCID}
}




\maketitle
\conf{\textit{  Proc. of International Conference on Artificial Intelligence, Computer, Data Sciences and Applications (ACDSA 2026) \\ 
5-7 February 2026, Boracay-Philippines}}
\begin{abstract}
    Methods for curbing the spread of misinformation in the Philippines remain inadequate. The internet as a medium for fake news necessitates fast, automated, and accessible countermeasures. A precursor study from 2020 benchmarks Transfer Learning (TL) techniques in building Filipino language fake news classifiers from a low-resource dataset. Despite promising results, the models from the aforementioned study cannot be easily deployed for a wide audience. In this work, we show that robust fake news classifiers for a morphologically rich language can be constructed from lightweight machine learning models and a low-resource dataset. We show that these machine learning models can be successfully deployed on a stringent infrastructure. First, we construct a dataset of Filipino language news articles. We extract Filipino linguistic features from the dataset. Next, we train a Logistic Regression model, which achieves an accuracy of 95\% without hyperparameter tuning. Using this model, we build and deploy a system for classifying Filipino language fake news.
\end{abstract}

%\copyrightnotice{XXX-X-XXXX-XXXX-X/XX/\$XX.00 ©20XX IEEE}




\begin{IEEEkeywords}
natural language processing, machine learning, fake news in Filipino language, Filipino linguistic features
\end{IEEEkeywords}

\section{Introduction}
Misinformation ranks among the world's top global risks as fake news outlets see traffic. In the Philippines, journalists and political analysts speculate that Former President Rodrigo Duterte's landslide victory in the 2016 presidential elections has been brought about by paid trolls disseminating fake news through social media outlets \cite{b1}. Former President Duterte earned publicity by popularizing a depiction of the Philippines as a \textit{narco-state}.  Though he insists that this rhetoric is truth, the UN Office on Drugs and Crime reports otherwise—the Philippines' drug use prevalence is lower than the global average \cite{b2}.

A previous work \cite{b3} explores Transfer Learning (TL) in building robust fake news classifiers for the morphologically rich Filipino language. Despite promising results, the deployment of the models (e.g., BERT) used in this work costs somewhere around 50,000 to 1,600,000 USD \cite{b4}. Hence, they cannot be easily deployed for consumer-level applications \cite{b3, b4}.

This work investigates lightweight machine learning models in tackling the classification of Filipino language fake news. A low-resource dataset of Filipino language news articles (Fake News Filipino 2024) is sourced from the internet to alleviate resource scarcity. A Logistic Regression model is trained on the combined Fake News Filipino and Fake News Filipino 2024 dataset, achieving an accuracy of 95\%. The authors successfully deploy the fake news classifier as a web extension (dubbed as FaKe) on a stringent infrastructure.

\section{Related Work}

Machine learning models have shown promise in automating fake news detection. A study \cite{b5} reports that Naive Bayes with \textit{n}-grams achieves 93\% accuracy on English datasets, while another study \cite{b6} finds that Support Vector Machine (SVM) reaches 96\% accuracy on a large corpora of about 20,000 articles. However, these studies focus on resource-rich English datasets.

For Filipino language news articles, a study \cite{b3} pioneers fake news detection using Byte-Pair Encoding (BPE) tokenization to handle morphologically-rich language features and out-of-vocabulary words. While another study \cite{b7} identifies 76 linguistic features for fake news detection in the Filipino context, their corpus used English-language articles. Two studies \cite{b8, b9} develop Filipino-specific readability metrics and linguistic features including traditional features (word count, sentence count), syllabic patterns based on Philippine orthography, and morphological features from verb inflections. These studies identify polysyllable word count, sentence count, and average sentence length as top predictors for Filipino text readability.

Despite advances, Filipino linguistic features have not been applied in training Filipino language fake news classifiers. Moreover, deployment costs for complex models remain prohibitive (\$50,000-\$1,600,000) \cite{b4}, necessitating the investigation of lightweight models for Filipino language fake news detection.

\section{Prepare Your Paper Before Styling}
Before you begin to format your paper, first write and save the content as a 
separate text file. Complete all content and organizational editing before 
formatting. Please note sections \ref{AA}--\ref{SCM} below for more information on 
proofreading, spelling and grammar.

Keep your text and graphic files separate until after the text has been 
formatted and styled. Do not number text heads---{\LaTeX} will do that 
for you.

\subsection{Abbreviations and Acronyms}\label{AA}
Define abbreviations and acronyms the first time they are used in the text, 
even after they have been defined in the abstract. Abbreviations such as 
IEEE, SI, MKS, CGS, ac, dc, and rms do not have to be defined. Do not use 
abbreviations in the title or heads unless they are unavoidable.

\subsection{Units}
\begin{itemize}
\item Use either SI (MKS) or CGS as primary units. (SI units are encouraged.) English units may be used as secondary units (in parentheses). An exception would be the use of English units as identifiers in trade, such as ``3.5-inch disk drive''.
\item Avoid combining SI and CGS units, such as current in amperes and magnetic field in oersteds. This often leads to confusion because equations do not balance dimensionally. If you must use mixed units, clearly state the units for each quantity that you use in an equation.
\item Do not mix complete spellings and abbreviations of units: ``Wb/m\textsuperscript{2}'' or ``webers per square meter'', not ``webers/m\textsuperscript{2}''. Spell out units when they appear in text: ``. . . a few henries'', not ``. . . a few H''.
\item Use a zero before decimal points: ``0.25'', not ``.25''. Use ``cm\textsuperscript{3}'', not ``cc''.)
\end{itemize}

\subsection{Equations}
Number equations consecutively. To make your 
equations more compact, you may use the solidus (~/~), the exp function, or 
appropriate exponents. Italicize Roman symbols for quantities and variables, 
but not Greek symbols. Use a long dash rather than a hyphen for a minus 
sign. Punctuate equations with commas or periods when they are part of a 
sentence, as in:
\begin{equation}
a+b=\gamma\label{eq}
\end{equation}

Be sure that the 
symbols in your equation have been defined before or immediately following 
the equation. Use ``\eqref{eq}'', not ``Eq.~\eqref{eq}'' or ``equation \eqref{eq}'', except at 
the beginning of a sentence: ``Equation \eqref{eq} is . . .''

\subsection{\LaTeX-Specific Advice}

Please use ``soft'' (e.g., \verb|\eqref{Eq}|) cross references instead
of ``hard'' references (e.g., \verb|(1)|). That will make it possible
to combine sections, add equations, or change the order of figures or
citations without having to go through the file line by line.

Please don't use the \verb|{eqnarray}| equation environment. Use
\verb|{align}| or \verb|{IEEEeqnarray}| instead. The \verb|{eqnarray}|
environment leaves unsightly spaces around relation symbols.

Please note that the \verb|{subequations}| environment in {\LaTeX}
will increment the main equation counter even when there are no
equation numbers displayed. If you forget that, you might write an
article in which the equation numbers skip from (17) to (20), causing
the copy editors to wonder if you've discovered a new method of
counting.

{\BibTeX} does not work by magic. It doesn't get the bibliographic
data from thin air but from .bib files. If you use {\BibTeX} to produce a
bibliography you must send the .bib files. 

{\LaTeX} can't read your mind. If you assign the same label to a
subsubsection and a table, you might find that Table I has been cross
referenced as Table IV-B3. 

{\LaTeX} does not have precognitive abilities. If you put a
\verb|\label| command before the command that updates the counter it's
supposed to be using, the label will pick up the last counter to be
cross referenced instead. In particular, a \verb|\label| command
should not go before the caption of a figure or a table.

Do not use \verb|\nonumber| inside the \verb|{array}| environment. It
will not stop equation numbers inside \verb|{array}| (there won't be
any anyway) and it might stop a wanted equation number in the
surrounding equation.

\subsection{Some Common Mistakes}\label{SCM}
\begin{itemize}
\item The word ``data'' is plural, not singular.
\item The subscript for the permeability of vacuum $\mu_{0}$, and other common scientific constants, is zero with subscript formatting, not a lowercase letter ``o''.
\item In American English, commas, semicolons, periods, question and exclamation marks are located within quotation marks only when a complete thought or name is cited, such as a title or full quotation. When quotation marks are used, instead of a bold or italic typeface, to highlight a word or phrase, punctuation should appear outside of the quotation marks. A parenthetical phrase or statement at the end of a sentence is punctuated outside of the closing parenthesis (like this). (A parenthetical sentence is punctuated within the parentheses.)
\item A graph within a graph is an ``inset'', not an ``insert''. The word alternatively is preferred to the word ``alternately'' (unless you really mean something that alternates).
\item Do not use the word ``essentially'' to mean ``approximately'' or ``effectively''.
\item In your paper title, if the words ``that uses'' can accurately replace the word ``using'', capitalize the ``u''; if not, keep using lower-cased.
\item Be aware of the different meanings of the homophones ``affect'' and ``effect'', ``complement'' and ``compliment'', ``discreet'' and ``discrete'', ``principal'' and ``principle''.
\item Do not confuse ``imply'' and ``infer''.
\item The prefix ``non'' is not a word; it should be joined to the word it modifies, usually without a hyphen.
\item There is no period after the ``et'' in the Latin abbreviation ``et al.''.
\item The abbreviation ``i.e.'' means ``that is'', and the abbreviation ``e.g.'' means ``for example''.
\end{itemize}
An excellent style manual for science writers is \cite{b7}.

\subsection{Authors and Affiliations}
\textbf{The class file is designed for, but not limited to, six authors.} A 
minimum of one author is required for all conference articles. Author names 
should be listed starting from left to right and then moving down to the 
next line. This is the author sequence that will be used in future citations 
and by indexing services. Names should not be listed in columns nor group by 
affiliation. Please keep your affiliations as succinct as possible (for 
example, do not differentiate among departments of the same organization).

\subsection{Identify the Headings}
Headings, or heads, are organizational devices that guide the reader through 
your paper. There are two types: component heads and text heads.

Component heads identify the different components of your paper and are not 
topically subordinate to each other. Examples include Acknowledgments and 
References and, for these, the correct style to use is ``Heading 5''. Use 
``figure caption'' for your Figure captions, and ``table head'' for your 
table title. Run-in heads, such as ``Abstract'', will require you to apply a 
style (in this case, italic) in addition to the style provided by the drop 
down menu to differentiate the head from the text.

Text heads organize the topics on a relational, hierarchical basis. For 
example, the paper title is the primary text head because all subsequent 
material relates and elaborates on this one topic. If there are two or more 
sub-topics, the next level head (uppercase Roman numerals) should be used 
and, conversely, if there are not at least two sub-topics, then no subheads 
should be introduced.

\subsection{Figures and Tables}
\paragraph{Positioning Figures and Tables} Place figures and tables at the top and 
bottom of columns. Avoid placing them in the middle of columns. Large 
figures and tables may span across both columns. Figure captions should be 
below the figures; table heads should appear above the tables. Insert 
figures and tables after they are cited in the text. Use the abbreviation 
``Fig.~\ref{fig}'', even at the beginning of a sentence.

\begin{table}[htbp]
\caption{Table Type Styles}
\begin{center}
\begin{tabular}{|c|c|c|c|}
\hline
\textbf{Table}&\multicolumn{3}{|c|}{\textbf{Table Column Head}} \\
\cline{2-4} 
\textbf{Head} & \textbf{\textit{Table column subhead}}& \textbf{\textit{Subhead}}& \textbf{\textit{Subhead}} \\
\hline
copy& More table copy$^{\mathrm{a}}$& &  \\
\hline
\multicolumn{4}{l}{$^{\mathrm{a}}$Sample of a Table footnote.}
\end{tabular}
\label{tab1}
\end{center}
\end{table}

\begin{figure}[htbp]
\centerline{\includegraphics{fig1.png}}
\caption{Example of a figure caption.}
\label{fig}
\end{figure}

Figure Labels: Use 8 point Times New Roman for Figure labels. Use words 
rather than symbols or abbreviations when writing Figure axis labels to 
avoid confusing the reader. As an example, write the quantity 
``Magnetization'', or ``Magnetization, M'', not just ``M''. If including 
units in the label, present them within parentheses. Do not label axes only 
with units. In the example, write ``Magnetization (A/m)'' or ``Magnetization 
\{A[m(1)]\}'', not just ``A/m''. Do not label axes with a ratio of 
quantities and units. For example, write ``Temperature (K)'', not 
``Temperature/K''.

\section*{Acknowledgment}

The preferred spelling of the word ``acknowledgment'' in America is without 
an ``e'' after the ``g''. Avoid the stilted expression ``one of us (R. B. 
G.) thanks $\ldots$''. Instead, try ``R. B. G. thanks$\ldots$''. Put sponsor 
acknowledgments in the unnumbered footnote on the first page.

\section*{References}

Please number citations consecutively within brackets \cite{b1}. The 
sentence punctuation follows the bracket \cite{b2}. Refer simply to the reference 
number, as in \cite{b3}---do not use ``Ref. \cite{b3}'' or ``reference \cite{b3}'' except at 
the beginning of a sentence: ``Reference \cite{b3} was the first $\ldots$''

Number footnotes separately in superscripts. Place the actual footnote at 
the bottom of the column in which it was cited. Do not put footnotes in the 
abstract or reference list. Use letters for table footnotes.

Unless there are six authors or more give all authors' names; do not use 
``et al.''. Papers that have not been published, even if they have been 
submitted for publication, should be cited as ``unpublished'' \cite{b4}. Papers 
that have been accepted for publication should be cited as ``in press'' \cite{b5}. 
Capitalize only the first word in a paper title, except for proper nouns and 
element symbols.

For papers published in translation journals, please give the English 
citation first, followed by the original foreign-language citation \cite{b6}.

\begin{thebibliography}{00}
\bibitem{b1} J. Lanuza, J. Ong, and R. Tapsell, ``Evolutions of 'fake news' from the south: Tracking disinformation innovations and interventions between the 2016 and 2019 Philippines elections,'' Berkman Klein Center for Internet \& Society, Harvard University, 2019. [Online]. Available: https://cyber.harvard.edu/sites/default/files/2019-11/Comparative\%20Approaches\%20to\%20Disinformation\%20-\%20Jose\%20Mari\%20Hall\%20Lanuza\%20Slides.pdf
\bibitem{b2} A. Yee, ``Post-truth politics and fake news in Asia,'' Global Asia, vol. 12, pp. 66--71, 2017.
\bibitem{b3} J. C. B. Cruz, J. A. Tan, and C. Cheng, ``Localization of fake news detection via multitask transfer learning,'' in Proc. 12th Language Resources and Evaluation Conf., 2020, pp. 2596--2604.
\bibitem{b4} A. Paleyes, R.-G. Urma, and N. D. Lawrence, ``Challenges in deploying machine learning: A survey of case studies,'' ACM Comput. Surv., vol. 55, no. 6, pp. 1--29, 2022.
\bibitem{b5} J. Y. Khan, Md. T. Khondaker, S. Afroz, G. Uddin, and A. Iqbal, ``A benchmark study of machine learning models for online fake news detection,'' Mach. Learn. Appl., vol. 4, p. 100032, 2021.
\bibitem{b6} D. Choudhury and T. Acharjee, ``A novel approach to fake news detection in social networks using genetic algorithm applying machine learning classifiers,'' Multimedia Tools Appl., vol. 82, no. 6, pp. 9029--9045, 2022.
\bibitem{b7} A. C. T. Fernandez and M. Devaraj, ``Computing the linguistic-based cues of fake news in the Philippines towards its detection,'' in Proc. 9th Int. Conf. Web Intelligence, Mining and Semantics, 2019, pp. 1--9.
\bibitem{b8} J. M. Imperial and E. Ong, ``Exploring hybrid linguistic feature sets to measure Filipino text readability,'' in Proc. Int. Conf. Asian Language Processing (IALP), 2020, pp. 175--180.
\bibitem{b9} J. M. Imperial and E. Ong, ``Diverse linguistic features for assessing reading difficulty of educational Filipino texts,'' arXiv:2108.00241, 2021.
\end{thebibliography}
\vspace{12pt}
\color{red}
IEEE conference templates contain guidance text for composing and formatting conference papers. Please ensure that all template text is removed from your conference paper prior to submission to the conference. Failure to remove the template text from your paper may result in your paper not being published.

\end{document}
