%   Filename    : chapter_6.tex
\chapter{Conclusion}

In this study, we constructed a balanced dataset of fake and authentic Filipino news articles named Fake News Filipino 2024, containing 1603 instances of real news and 1603 instances of fake news. We augmented Fake News Filipino and compiled a joint corpus for training several machine learning models. We compared the performance of four models: Logistic Regression, Multinomial Naive Bayes, Random Forest, and Support Vector Classifier in classifying news articles across three datasets: Fake News Filipino 2024, Fake News Filipino, and a joint corpus. Hyperparameter tuning was conducted to determine the optimal hyperparameters of each classifier. Logistic Regression emerged as the most suitable model for deployment as it boasted the highest accuracies, without requiring hyperparameter tuning.

\section{Limitations}

Discussed in detail in section \ref{extension-limitations}, the FaKe web extension bears several limitations owing to its present architecture. The discrepancies between the writing styles of the news articles in Fake News Filipino and Fake News Filipino 2024 may have impacted the performance of the classifiers on the joint corpus, as discussed in section \ref{dataset-limitation}. Robust Filipino language fake news detection remains a challenge, but this paper has made long strides in the proper direction.

\section{Recommendations}

For future studies, we recommend the expansion of the corpus, to include more domains and incorporate more writing styles. We recommend the development of a Filipino fake news detection system on a less stringent infrastructure. 