%   Filename    : chapter_5.tex
\chapter{Discussion}

As shown in the results of two-way ANOVA together with the post-hoc tests, three out of four classifiers, namely Logistic Regression, Random Forest, and Support Vector Classifier, does not have a signifcant change in performance between Filipino Fake News and Filipino Fake News dataset. However, all classfiers' accuracies are significantly lower in the combined dataset. The reason for this may be attributed to the difference in style of writing of news articles between the two datasets. This is supported by the difference in average OOV count, average stop words count, and average readability scores of the two datasets. Which means that when the two datasets are separate, the writing style is more consistent, therefore, the classifiers perform better. On the other hand, when the datasets are combined, the styles are now less consistent, affecting the accuracies of the classifiers. 

For the comparisons between classifiers within different datasets, it was shown that Logistic Regression and Support Vector Classifier does not have significant differences in their accuracies regardless of the dataset and that both classifers performs significantly better than Random Forest and Multinomial Naive Bayes. It must be noted, however, that hyperparameter tuning is requried for Support Vector Classfier for optimal performance whereas it is not for Logistic Regression since hyperparameter tuning does not have significant effect on the performance of Logistic Regression. Thus, it can be said that Logistic Regression is more efficient than Support Vector Classifier. Overall, the most suitable model for deployment in the system is Logistic Regression since it has best performance across all datasets without requiring hyperparam tuning.