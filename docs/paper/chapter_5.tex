%   Filename    : chapter_6.tex
\chapter{Conclusion}

In this study, we constructed a balanced dataset of fake and authentic Filipino news articles named Fake News Filipino 2024, containing 1603 instances of real news and 1603 instances of fake news. We augmented Fake News Filipino and compiled a joint corpus for training several machine learning models. We compared the performance of four classifiers: Logistic Regression, Multinomial Naive Bayes, Random Forest, and Support Vector Classifier in classifying news articles across three datasets: Fake News Filipino 2024, Fake News Filipino, and a joint corpus. Hyperparameter tuning was conducted to determine the optimal hyperparameters of each classifier. Logistic Regression emerged as the most suitable model for deployment as it boasted the highest accuracies, without requiring hyperparameter tuning. The application yielded uniform results across different browsers.

\section{Limitations}

As discussed in detail in Section \ref{extension-limitations}, the FaKe web extension bears several limitations owing to its present architecture. The discrepancies between the writing styles of the news articles in Fake News Filipino and Fake News Filipino 2024 may have impacted the performance of the classifiers on the joint corpus, as discussed in Section \ref{dataset-limitation}. Robust Filipino language fake news detection remains a challenge, but this study has made long strides in the proper direction.

\section{Recommendations}

For future studies, we recommend the expansion of the corpus, to include more domains and incorporate more writing styles. We recommend the development of a Filipino fake news detection system on a less stringent infrastructure. Furthermore, we recommend the undertaking of future studies that tackle other forms of fake news (e.g. spam, tweets, posts, etc.). With respect to deploying a microservice on Render, it is possible to remove the service downtime entirely by incurring a small monthly fee. Articles flagged by the web extension should have their sources collated into a database. Future studies may build a dataset that incorporates article headings and article sources as features. Lastly, as Filipino linguistic features are in a stage of infancy, more studies should be conducted in this domain.