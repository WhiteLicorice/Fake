%   Filename    : abstract.tex 
\begin{abstract}
To date, advances in Filipino language fake news detection remain inadequate in stemming the rampant spread of fake news in the Philippine information economy. A precursor study in 2020 benchmarks deep learning techniques in building robust Filipino language fake news classifiers from a low-resource dataset and for a morphologically rich language. Despite promising results, the models from the study cannot be feasibly deployed due to unreasonable cost of deployment, thus being incapable of utilization by the public. In this paper, we propose the investigation of machine learning models for classifying fake news in the Filipino language. Our eventual goal is the deployment of the most suitable model as a cross-browser web extension for detecting Filipino language fake news. We set out to train machine learning models on a dataset comprised of news articles written mostly in Filipino with a few terms in the English vernacular. The dataset will be perfectly balanced between instances of fake news and authentic news.

\begin{comment}
    Finish abstract once results are in.
\end{comment}

%  Do not put citations or quotes in the abstract.

\begin{flushleft}
\begin{tabular}{lp{4.25in}}
\hspace{-0.5em}\textbf{Keywords:}\hspace{0.25em} & machine learning, natural language processing, fake news, filipino language, filipino linguistic features, classifiers, feature extraction, hyperparameter tuning, low-resource dataset, naive bayes, logistic regression, random forest, svc
\end{tabular}
\end{flushleft}
\end{abstract}

