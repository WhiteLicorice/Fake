%   Filename    : abstract.tex

\begin{center}
    \textbf{Abstract}
    \end{center}
    \setlength{\parindent}{0pt}
    Present methods for curbing the rampant spread of misinformation in the Philippines remain inadequate. The internet as a medium for spreading fake news necessitates fast, automated tools as countermeasures accessible to the public. A precursor study from 2020 benchmarks deep learning techniques in building Filipino language fake news classifiers from a low-resource dataset. Despite promising results, the models from the previous study cannot be feasibly deployed for a wide audience due to unreasonable cost of deployment. In this work, we make several contributions. First, we construct a dataset of Filipino language news articles, alleviating resource scarcity, and name this dataset Fake News Filipino 2024. Next, we apply techniques to extract Filipino linguistic features from our corpus. We show that robust fake news classifiers for a morphologically rich language may be constructed from simple machine learning models and a low-resource dataset. We perform hyperparameter tuning and benchmark these classifiers across three datasets comprised of Filipino news articles. Our best model, Logistic Regression, achieves a best accuracy of 95\% without hyperparameter tuning. Lastly, we build a cross-browser extension for classifying Filipino language fake news, deploying a trained Logistic Regression model on a stringent infrastructure. The FaKe web extension is available to the public on this paper's code repository.


    \begin{tabular}{lp{4.25in}}
    \hspace{-0.5em}\textbf{Keywords:}\hspace{0.25em} & machine learning, natural language processing, fake news in Filipino language, Filipino linguistic features
    \\
\end{tabular}

