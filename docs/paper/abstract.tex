%   Filename    : abstract.tex 

\begin{center}
    \textbf{Abstract}
    \end{center}
    \setlength{\parindent}{0pt}
    Present methods for curbing the rampant spread of misinformation in the Philippines remain inadequate. The internet used as a medium for spreading fake news necessitates fast, automated computational tools as countermeasures accessible to the public. A precursor study from 2020 benchmarks deep learning techniques in building Filipino language fake news classifiers from a low-resource dataset. Despite promising results, the models from the previous study cannot be feasibly deployed for a wide audience due to unreasonable cost of deployment. In this work, we make several contributions. First, we construct a second dataset of Filipino language news articles, alleviating resource scarcity, and name this dataset Fake News Filipino 2024. Next, we apply bleeding edge techniques to extract Filipino linguistic features from our corpus, and use these features in training and benchmarking machine learning models. We show that robust fake news classifiers for a morphologically rich language may be constructed from simple machine learning models and a low-resource dataset. We benchmark these machine learning models across three datasets comprised of Filipino news articles and perform hyperparameter tuning. Our best model, Logistic Regression, achieves an accuracy of 93\% without hyperparameter tuning. Hyperparameter tuning allows it to achieve a 95\% accuracy. Lastly, we build a cross-browser web extension for classifying Filipino language fake news, deploying a trained Logistic Regression model on a stringent infrastructure. The FaKe web extension has been made available to the public on this paper's code repository.
    
    
    \begin{tabular}{lp{4.25in}}
    \hspace{-0.5em}\textbf{Keywords:}\hspace{0.25em} & machine learning, natural language processing, fake news, journalism, computers in other domains, filipino language, filipino linguistic features, classifiers, feature extraction, hyperparameter tuning, low-resource dataset, naive bayes, logistic regression, random forest, svc\\
\end{tabular}
    
    