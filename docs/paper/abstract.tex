%   Filename    : abstract.tex 
\begin{abstract}

    Present methods for curbing the rampant spread of misinformation in the Philippines remain inadequate. The internet used as a medium for spreading fake news necessitates fast, automated computational tools as countermeasures accessible to the public. A precursor study from 2020 benchmarks deep learning techniques in building Filipino language fake news classifiers from a low-resource dataset. Despite promising results, the models from the previous study cannot be feasibly deployed for a wide audience due to unreasonable cost of deployment. In this work, we make several contributions. First, we construct a second dataset of Filipino language news articles, alleviating resource scarcity, and name this dataset \enquote{Fake News Filipino 2024}. Next, we apply bleeding edge techniques to extract Filipino linguistic features, and use these features in training and benchmarking machine learning models. We show that robust fake news classifiers for a morphologically rich language may be constructed from simple machine learning models and a low-resource dataset. Our best model and feature space achieves an accuracy of 97\% on the corpus. Lastly, we build a cross-browser web extension for classifying Filipino language fake news using our best model. The \enquote{FaKe} web extension has been made available to the public on this paper's code repository.
    
    \begin{comment}
        Finalize abstract once tuned results from Chip are in.
    \end{comment}
    
    %  Do not put citations or quotes in the abstract.
    
    \begin{flushleft}
    \begin{tabular}{lp{4.25in}}
    \hspace{-0.5em}\textbf{Keywords:}\hspace{0.25em} & machine learning, natural language processing, fake news, journalism, computers in other domains, filipino language, filipino linguistic features, classifiers, feature extraction, hyperparameter tuning, low-resource dataset, naive bayes, logistic regression, random forest, svc
    \end{tabular}
    \end{flushleft}
    \end{abstract}
    
    